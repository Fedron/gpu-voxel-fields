\chapter{Conclusion}
A final discussion on the results of the findings in this paper, and the improvements and further work that could be
carried out in the future.

\section{Reflection}
This paper has found an algorithm for generating distance fields from a voxel grid for use in a dynamic world; being
the primary objective of this paper we can say it has been a success. Other methods for generating distance fields
were also discussed and the differences between them, and the potential for performance improvements was also discussed
as various improvements and optimizations were applied over the course of implementation. A deep understanding of
distance fields and the algorithms used to generate them was developed, as well as a detailed guide for how those
algorithms work and their limitations that could serve as a template, or starting point, for others.

A visual demonstration application, in the form of a voxel ``painting'' game, was developed that allowed for a
testing framework to be developed to accurately test the performance of various algorithms. It also served as a way
of identifying whether the algorithms being used created an accurate distance field without artifacts. While the FPS of
the demonstration application suffers at very large world sizes, it serves as a solid base from which a more complex
voxel driven game could be developed using distance fields for rendering.

The rendering of distance fields, and the memory intensity of them, was not entirely understood before undertaking this
work; the end result was the discovery that a naive ray marcher becomes a significant bottleneck due to the large amount
of data the GPU needs to have read access to and the cost of binding that data to the ray marching compute shader.

\section{Further Work}
As previously discussed, the distance fields, and raw voxel grid, require a large amount of memory; sparse voxel octrees
are a memory efficient way to represent voxel grids but as discovered in this paper updating a distance field is more
performant. It would be beneficial to identify a way to improve the memory efficiency of the distance field
representation, this could include:

\begin{itemize}
      \item A hybrid SVO and distance field approach allowing for sparse representation in areas where high fidelity is
            not required.
      \item Distance field compression, or run-length encoding. Similar to how SVOs achieve their high compression values,
            a distance field may be able to be compressed to reduce its memory footprint.
\end{itemize}

The ray marching of several large distance fields was also identified as a bottleneck in this paper; however, this could
be remedied by improving the representation by using the suggested approaches above for improving the memory efficiency
of the distance field. Alternatively, a better suited ray marching approach could be implemented instead as there are
various techniques for only casting rays that have high significance that are already used in ray casting through a
voxel world such as cone tracing.
