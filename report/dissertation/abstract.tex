\chapter*{Abstract}

This dissertation investigates the feasibility of generating distance fields in real-time on modern GPUs, addressing a
critical performance bottleneck in dynamic graphical applications. While distance fields offer exceptional rendering
efficiency, their adoption in interactive environments has been limited by the computational cost of their generation
when geometry changes.

The research employs a systematic comparative analysis of distance field generation techniques including brute force
computation, spatial chunking, the Fast Iterative Method, and the Jump Flooding Algorithm. Performance evaluation
criteria encompass computational throughput, memory requirements, and output quality across diverse test cases. Results
demonstrate that existing single-method approaches present inherent limitations for real-time applications. In response,
this dissertation proposes a novel hybrid algorithm that strategically combines the convergence characteristics of the
Fast Iterative Method with the parallel efficiency of the Jump Flooding Algorithm, enhanced by an adaptive
level-of-detail mechanism that optimizes computational resources.

Experimental validation confirms that the proposed approach achieves significant performance improvements over
traditional methods while maintaining necessary accuracy for real-time applications. The findings establish that dynamic
distance field generation is indeed viable for interactive applications, offering a more efficient alternative to other
real-time volumetric representation techniques. This research contributes to the field by providing a practical,
implementation-ready solution for incorporating distance fields in applications where geometry undergoes frequent
modification, potentially expanding their utility in interactive simulations, procedural content generation, and
advanced rendering pipelines.

\begin{center}
    \textit{The implementation code and demonstration application used in this research are available at: \url{https://github.com/Fedron/gpu-voxel-fields}}
\end{center}
