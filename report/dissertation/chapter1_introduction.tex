\chapter{Introduction}

\section{Problem Statement}
Rendering large-scale voxel worlds with dynamic scenes presents significant challenges due to the vast amount of data
required to to represent 3D environments. Voxel-based rendering, where the scene is composed of volumetric pixels (voxels),
is an attractive alternative to traditional triangle-based rendering for its ability to represent complex geometry and
volumetric effects; however, the scalability of voxel-based rendering is limited by the amount of memory required to store
the voxel data, and often times the computational cost of updating hyper-compressed voxel data structures.

Sparse Voxel Octrees (SVOs)~\cite{Laine_Karras_2010} are a popular data structure for representing voxel scenes, as they
provide a compact representation of the scene by storing only the occupied voxels in a tree structure. Octrees, or nodes,
can be dynamically subdivided to increase the resolution of the scene, and provide a straighforward way of introducting
Level of Detail (LOD). Octrees are also well suited to ray tracing, as they provide a natural way of separating a scene
into bounding boxes that can be quickly tested for intersection~\cite{Ize_2009}. SVOs allow for high compression ratios,
and performant ray tracing, for the rendering of large (typically in the millions) of voxels.

SVOs are not without their limitations, however. Updating an SVO can be a straightforward process as only the affected
nodes need updating; however, an SVO is usually further compressed into a format better suited for GPU rendering, using
techniques such as run-length encoding (RLE)~\cite{Eisenwave_RLE} and space filling curves (SFC)~\cite{Eisenwave_SFC}.
Updating these compressed formats, and transferring the new data to the GPU, can be computationally expensive~\cite{Crassin_2012}.
This makes SVOs less suitable for dynamic scenes where voxels can be added, removed, or modified frequently.

\section{Aims and Objectives}
The primary goal of this dissertation is to investigate the feasibility, and identify potential techniques and approaches
that could improve the performance of using SVOs for rendering large-scale dynamic voxel scenes. The aims of this
dissertation can be broken down into the following objectives:

\begin{enumerate}
    \item Investigate current techniques for constructing and rendering SVOs.
    \item Design a system for updating SVOs, on the GPU, to allow for large dynamic voxel scenes.
    \item Develop a renderer, using Vulkan, that addresses the challenges of data transfer between the host and device of large SVOs.
    \item Evaluate the performance of the system, and identify potential areas for improvement. See Section~\ref{sec:performance_metrics}.
    \item Investigate potential techniques for further improving the performance of SVOs for dynamic scenes.
\end{enumerate}

\subsection{Performance Metrics} \label{sec:performance_metrics}
The performance of the system will be evaluated using the following metrics:

\begin{description}
    \item[Frames Per Second (FPS)] The number of frames rendered per second. At a minimum for real-time rendering, and for
        interactive applications such as games, the system should be able to render at a consistent 30 FPS.

        Since FPS isn't always a good indicator of performance, frame time should also be considered and is measured as the
        time taken to render a single frame including the update and render time. For a consistent 30 FPS, the frame time should
        be an average of 33.33ms.
    \item[Memory Usage] The amount of memory used by the SVO on the GPU. The amount of memory an SVO uses is heavily dependant
        on the resolution of the scene, and the exact compression techniques used. For a 512\textsuperscript{3} voxel scene,
        with a 9 level SVO, the memory usage should be less than 1GB~\cite{Crassin_2012,Laine_Karras_2010}.
    \item[Data Transfer Time] The time taken to transfer the updated SVO data between the host and device. TODO
    \item[Construction and Update Time] The time taken to update the SVO on the GPU. Using the same 512\textsuperscript{3} voxel
        scene, the construction time should be less than 100ms~\cite{Crassin_2012,Laine_Karras_2010}; the construction of
        an SVO includes updating the tree structure with new voxels.
    \item[Rendered Voxels] The number of voxels rendered in the scene. A 512\textsuperscript{3} voxel scene consits of
        134,217,728 voxels; having a fully populated octree would not be feasible and would mean a majority of the voxels are
        obsured. Using a suitable voxel scene, at least 1,000,000 visible voxels should be rendered at the 30 FPS target.
\end{description}

\section{Scope and Limitations}
A fully-featured voxel renderer is out of scope for this dissertation, this includes features such as:

\begin{description}
    \item[Lighting and shading effects] Effects such as ambient occlusion, refractions, reflections, and global illumination
        are not considered in this dissertation as they are renderer specific and not the focus of this dissertation.
    \item[Transparency and volumetric effects] Rendering transparent voxels, or volumetric effects such as fog, smoke, or
        fire, add additional complexity to the rendering process, which is not the focus of this dissertation.
    \item[Identifying new techniques for SVO compression] The focus of this dissertation is on the performance of updating
        SVOs, not on the compression techniques used to store the SVO on the GPU.
    \item[Optimizing the rendering engine] A simple Vulkan renderer will be built to demonstrate the performance of the
        SVO update system, but the focus will be on the SVO update system itself.
\end{description}

\section{Thesis Outline}

\begin{enumerate}
    \item \textbf{Introduction} - A brief introduction to the problem statement, aims and objectives, and the scope and limitations
          of the dissertation.
    \item \textbf{Literature Review} - A review and discussion of the current techniques used for rendering voxel scenes,
          with a focus on Sparse Voxel Octrees; and the history of voxel rendering.
    \item \textbf{System Design} - A detailed explanation of the system design, including the data structures used, the
          algorithms for updating the SVO, and the renderer design.
    \item \textbf{Development} - A more in-depth look at the implementation of the system, as set out in Section~\ref{chapter:system_design}.
    \item \textbf{Testing and Evaluation} - A discussion on the performance of the system, and the results of the evaluation
          against the metrics set out in Section~\ref{sec:performance_metrics}.
    \item \textbf{Conclusion} - A summary of the dissertation, including the findings, limitations, and potential future work.
\end{enumerate}