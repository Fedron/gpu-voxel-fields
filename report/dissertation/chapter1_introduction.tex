\chapter{Introduction}
In recent years, real-time computer graphics applications have increasingly adopted distance fields as a fundamental
representation for rendering and physics simulations~\cite{jones2001using}. Distance fields, which encode the minimum
distance from any point to the nearest surface, provide an elegant solution for various graphics operations including
collision detection~\cite{fuhrmann2003distance}, soft shadows~\cite{tan2022rtsdf}, and ambient
occlusion~\cite{wright2015dynamic}. While techniques exist for generating distance fields in real-time from a triangle
mesh~\cite{lou2023dynamic}, techniques covering distance field generation from discrete voxel data are uncommon.

\section{Problem Statement}
Current approaches to distance field generation from voxel grids present various tradeoffs that limit their
effectiveness in dynamic scenes. The Jump Flooding Algorithm (JFA)~\cite{rong2006jump,rong2007variants,wang2023prf},
while efficient for parallel computation, introduces accuracy issues particularly at larger distances from surfaces and
near feature edges. Scan, or prefix sum-based, approaches~\cite{erleben2008signed} provide accurate results but suffer
from inherent sequential dependencies that limit GPU parallelization. Wavefront propagation methods can efficiently
update local regions but may struggle with concurrent updates in complex scenes~\cite{teodoro2013efficient}.

Common optimization strategies, such as spatial partitioning into smaller chunks for localized
updates~\cite{naylor1992interactive}, introduce their own challenges including boundary artifacts and increased memory
management overhead. While these techniques work well in isolation for specific use cases, there remains a fundamental
gap in solutions that can handle arbitrary dynamic scene modifications while maintaining both accuracy and performance.
This research investigates whether a novel hybrid approach—combining elements of existing techniques or developing new
algorithmic patterns—could better address these challenges.

\section{Aims and Objectives} \label{sec:aims_obj}
This research aims to develop and evaluate novel GPU-based techniques for rapid distance field generation from voxel
grid representations. The primary objectives are:

\begin{itemize}
  \item To analyze and classify existing approaches to distance field generation, with particular focus on
        GPU-accelerated methods.
  \item To develop new algorithms that optimize the conversion process from voxel grids to distance fields.
  \item To implement and validate these algorithms on modern GPU architectures.
  \item To establish a comprehensive comparison framework for evaluating different distance field generation techniques.
\end{itemize}

\subsection{Performance Metrics}
The evaluation of the proposed methods will be conducted against sparse voxel octree implementations, which currently
represent the state-of-the-art in many graphics applications. Key performance metrics include:

\begin{enumerate}
  \item Computation time for initial distance field generation.
  \item Memory consumption during generation and storage.
  \item Update latency for localized geometric changes.
  \item Scalability with increasing voxel grid resolution.
  \item Accuracy of distance field values compared to analytical solutions.
  \item GPU resource utilization, including memory bandwidth and compute occupancy.
\end{enumerate}

\section{Scope and Limitations}
While this research addresses the core challenge of distance field generation, several related aspects fall outside its
scope:

\begin{itemize}
  \item The optimization of ray marching techniques for distance field rendering.
  \item The development of new compression methods for distance field storage.
  \item The optimization of the underlying renderer, which will include features like:

        \begin{itemize}
          \item Memory management between the CPU and GPU.\@
          \item Sychronization with the windowing framework.
          \item Optimizing presentation of ray marching output image.
        \end{itemize}
\end{itemize}

The focus remains specifically on the GPU-based generation process and its performance characteristics in dynamic
scenarios. The research assumes access to modern GPU hardware and primarily targets real-time graphics applications
where frequent distance field updates are required.

A sample voxel painting graphics application will be implemented that will serve as a demonstration and benchmark for
real-time performance of the distance filed regeneration. For benchmarking the application will be slightly altered to
allow for automatic updates to the world, and reproducibility of test results; a detailed explanation of this
application can be found in Section~\ref{sec:demo_app}.
